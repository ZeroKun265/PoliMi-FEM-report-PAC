\documentclass{Configuration_Files/PoliMi3i_thesis}

%------------------------------------------------------------------------------
%	REQUIRED PACKAGES AND  CONFIGURATIONS
%------------------------------------------------------------------------------

% CONFIGURATIONS
\usepackage{parskip} % For paragraph layout
\usepackage{setspace} % For using single or double spacing
\usepackage{emptypage} % To insert empty pages
\usepackage{multicol} % To write in multiple columns (executive summary)
\setlength\columnsep{15pt} % Column separation in executive summary
\setlength\parindent{0pt} % Indentation
\raggedbottom  

% PACKAGES FOR TITLES
\usepackage{titlesec}
% \titlespacing{\section}{left spacing}{before spacing}{after spacing}
\titlespacing{\section}{0pt}{3.3ex}{2ex}
\titlespacing{\subsection}{0pt}{3.3ex}{1.65ex}
\titlespacing{\subsubsection}{0pt}{3.3ex}{1ex}
\usepackage{color}

% PACKAGES FOR LANGUAGE AND FONT
\usepackage[italian]{babel} % The document is in English  
\usepackage[utf8]{inputenc} % UTF8 encoding
\usepackage[T1]{fontenc} % Font encoding
\usepackage[11pt]{moresize} % Big fonts

% PACKAGES FOR IMAGES
\usepackage{graphicx}
\usepackage{transparent} % Enables transparent images
\usepackage{eso-pic} % For the background picture on the title page
\usepackage{subfig} % Numbered and caption subfigures using \subfloat.
\usepackage{tikz} % A package for high-quality hand-made figures.
\usetikzlibrary{}
\graphicspath{{./Images/}} % Directory of the images
\usepackage{caption} % Coloured captions
\usepackage{xcolor} % Coloured captions
\usepackage{amsthm,thmtools,xcolor} % Coloured "Theorem"
\usepackage{float}

% STANDARD MATH PACKAGES
\usepackage{amsmath}
\usepackage{amsthm}
\usepackage{amssymb}
\usepackage{amsfonts}
\usepackage{bm}
\usepackage[overload]{empheq} % For braced-style systems of equations.
\usepackage{fix-cm} % To override original LaTeX restrictions on sizes

% PACKAGES FOR TABLES
\usepackage{tabularx}
\usepackage{longtable} % Tables that can span several pages
\usepackage{colortbl}

% PACKAGES FOR ALGORITHMS (PSEUDO-CODE)
\usepackage{algorithm}
\usepackage{algorithmic}

% PACKAGES FOR REFERENCES & BIBLIOGRAPHY
\usepackage[colorlinks=true,linkcolor=black,anchorcolor=black,citecolor=black,filecolor=black,menucolor=black,runcolor=black,urlcolor=black]{hyperref} % Adds clickable links at references
\usepackage{cleveref}
\usepackage[square, numbers, sort&compress]{natbib} % Square brackets, citing references with numbers, citations sorted by appearance in the text and compressed
\bibliographystyle{abbrvnat} % You may use a different style adapted to your field

% OTHER PACKAGES
\usepackage{pdfpages} % To include a pdf file
\usepackage{afterpage}
\usepackage{lipsum} % DUMMY PACKAGE
\usepackage{fancyhdr} % For the headers
\fancyhf{}

% Input of configuration file. Do not change config.tex file unless you really know what you are doing. 
\input{Configuration_Files/config}

%----------------------------------------------------------------------------
%	NEW COMMANDS DEFINED
%----------------------------------------------------------------------------

% EXAMPLES OF NEW COMMANDS
\newcommand{\bea}{\begin{eqnarray}} % Shortcut for equation arrays
\newcommand{\eea}{\end{eqnarray}}
\newcommand{\e}[1]{\times 10^{#1}}  % Powers of 10 notation

%----------------------------------------------------------------------------
%	ADD YOUR PACKAGES (be careful of package interaction)
%----------------------------------------------------------------------------
\usepackage{tikz}
\usepackage{circuitikz}
\usepackage{subcaption}

%----------------------------------------------------------------------------
%	ADD YOUR DEFINITIONS AND COMMANDS (be careful of existing commands)
%----------------------------------------------------------------------------

%----------------------------------------------------------------------------
%	BEGIN OF YOUR DOCUMENT
%----------------------------------------------------------------------------

\begin{document}

\fancypagestyle{plain}{%
\fancyhf{} % Clear all header and footer fields
\fancyhead[RO,RE]{\thepage} %RO=right odd, RE=right even
\renewcommand{\headrulewidth}{0pt}
\renewcommand{\footrulewidth}{0pt}}

%----------------------------------------------------------------------------
%	TITLE PAGE
%----------------------------------------------------------------------------

\pagestyle{empty} % No page numbers
\frontmatter % Use roman page numbering style (i, ii, iii, iv...) for the preamble pages

\puttitle{
	title=Raccolta report di esperienze su Abaqus, % Title of the thesis
	course=086510 Progettazione Assistita dal Calcolatore,
	academicyear={2025-26}, 
}

%----------------------------------------------------------------------------
%	PREAMBLE PAGES: ABSTRACT (inglese e italiano), EXECUTIVE SUMMARY
%----------------------------------------------------------------------------
\startpreamble
\setcounter{page}{1} % Set page counter to 1

\chapter{Sommario}
Questo documento è l'unione di 3 report differenti, svolti durante il corso di Progettazione Assistita al Calcolatore:
\begin{itemize}
    \item Banchetto di prova SHM.
    \item Mensola 2D incastrata con intagli.
    \item Recipiente in pressione.
\end{itemize}

Per ogni report verrà descritto il problema reale da studiare, le scelte di modellazione eseguite e, dove applicabile, risoluzioni analitiche.

% TABLE OF CONTENTS
\thispagestyle{empty}
\tableofcontents % Table of contents 
\thispagestyle{empty}
\cleardoublepage

%-------------------------------------------------------------------------
%	THESIS MAIN TEXT
%-------------------------------------------------------------------------
% In the main text of your thesis you can write the chapters in two different ways:
%
%(1) As presented in this template you can write:
%    \chapter{Title of the chapter}
%    *body of the chapter*
%
%(2) You can write your chapter in a separated .tex file and then include it in the main file with the following command:
%    \chapter{Title of the chapter}
%    \input{chapter_file.tex}
%
% Especially for long thesis, we recommend you the second option.

\addtocontents{toc}{\vspace{2em}} % Add a gap in the Contents, for aesthetics
\mainmatter % Begin numeric (1,2,3...) page numbering

% --------------------------------------------------------------------------
% NUMBERED CHAPTERS % Regular chapters following
% --------------------------------------------------------------------------
\chapter{Banchetto SHM}
\section{Problema reale}
\label{ch:problema_reale}%
\begin{figure}[H]
    \centering

    \begin{minipage}{0.3\linewidth}
        \centering
        \includegraphics[width=\linewidth]{Images/carr.png}
        \caption{Carrello a terra}
        \label{fig:carr}
    \end{minipage}
    \hfill
    \begin{minipage}{0.3\linewidth}
        \centering
        \includegraphics[width=\linewidth]{Images/Banchetto prova.png}
        \caption{Banco di prova con sistema di carico}
        \label{fig:banco}
    \end{minipage}
    \hfill
    \begin{minipage}{0.3\linewidth}
        \centering
        \includegraphics[width=\linewidth]{Images/cern.png}
        \caption{Cerniera a terra}
        \label{fig:cern}
    \end{minipage}

\end{figure}

Nel seguente rapporto, si vogliono studiare le deformazioni di una struttura a travi di un banco di prova, rappresentata nella figura \ref{fig:banco}.
Si modellano le aste con elementi truss lineari (T2D2) e elementi beam lineari (B21) utilizzando il risolutore di simulazioni Abaqus e adottando un approccio analitico (solo per i truss).

Le 7 aste sono collegate tra di loro con delle piastre bullonate, mentre la parte sinistra inferiore del telaio è vincolato a terra con una cerniera, quella superiore da un carrello. Un carico di $428 \space N$ viene applicato dal sistema di carico a destra.

\subsection{Ottenimento dei vincoli}
\label{ss:ottenimento_vincoli}%
Le aste sono vincolate con delle piastre di giunzione bullonate, come in figura \ref{fig:giunz}.
Queste possono agire da vincolo di incastro quando ai bulloni viene applicata una coppia di serraggio di $10 \space Nm$, oppure da vincolo di cerniera se non sono serrati.

Si ottengono i vincoli di carrello e cerniera a terra con due montaggi diversi, rispettivamente mostrati in figura \ref{fig:carr} e figura \ref{fig:cern}


\subsection{Travi e giunzioni}
\label{ss:travi_giunzioni}%
\begin{figure}[H]
    \centering

    \begin{minipage}{0.40\linewidth}
        \centering
        \includegraphics[width=0.6\linewidth]{Images/giunzione.png}
        \caption{Piastra di giunzione in attivo}
        \label{fig:giunz}
    \end{minipage}
    \hfill
    \begin{minipage}{0.40\linewidth}
        \centering
        \includegraphics[width=0.6\linewidth]{Images/Piastra tavola.png}
        \caption{Quote della piastra di giunzione}
        \label{fig:giunz_tavola}
    \end{minipage}

\end{figure}

Le varie dimensioni della piastra sono rappresentate nella figura \ref{fig:giunz_tavola}. I bulloni utilizzati sono M6.
Nella figura \ref{fig:asta_giunta} si ha la rappresentazione del sistema di fissaggio delle aste.
\begin{figure}[H]
    \centering
    \includegraphics[width=0.50\linewidth]{Images/sist fissaggio.png}
    \caption{Sistema di fissaggio delle aste}
    \label{fig:asta_giunta}
\end{figure}

Le travi sono in alluminio, con il modulo di Young $E = 61639.8 MPa$, sono di sezione quadrata cava, con le quote riportate in figura \ref{fig:aste_quote}


\begin{figure}[H]
    \centering
    \includegraphics[width=1\linewidth]{Images/asta_quote.png}
    \caption{Quote di asta e sezione trasversale}
    \label{fig:aste_quote}
\end{figure}

\subsection{Posizionamento e utilizzo degli estensimetri}
\label{s:estensimetri}%
Si posizionano degli estensimetri sulle aste della struttura, collegati a diversi canali, numerati da 0 a 7, connessi poi a una scheda di acquisizione \textit{National Instrument NI-9235} con software \textit{LabVIEW}.
Ogni estensimetro è di tipo \textit{1-LY13-6/120} , specifico per leghe di alluminio, montati come nell'immagine \ref{fig:est_montaggio}. Il circuito di misura è rappresentato nella figura \ref{fig:schema_wheatstone}

\begin{figure}[H]
    \centering
    \input{ponte}
\end{figure}

Gli estensimetri sono posizionati, uno per asta, come nell'immagine \ref{fig:posiz_est}


\begin{figure}[H]
    \centering

    \begin{minipage}{0.45\linewidth}
        \centering
        \includegraphics[width=\linewidth]{Images/Montaggio estensimetro.png}
        \caption{Montaggio estensimetro}
        \label{fig:est_montaggio}
    \end{minipage}
    \hfill
    \begin{minipage}{0.45\linewidth}
        \centering
        \includegraphics[width=0.65\linewidth]{Images/posizioni_est.png}
        \caption{Posizioni degli estensimetri}
        \label{fig:posiz_est}
    \end{minipage}

\end{figure}

I dettagli della configurazione corrente sono:
\begin{itemize}
    \item Gli estensimetri misurano uno alla volta
    \item Gli effetti termici sull'estensimetro non sono compensati
    \item Le resistenze dei fili non sono compensate
\end{itemize}

\section{Modellazione con elementi truss}
\label{ch:mod_truss}%
Si studia ora la struttura con elementi lineari truss: due nodi e forze solo assiali

\subsection{Trattazione analitica}
\label{s:truss_analog}%
\begin{figure}[H]
    \centering
    \includegraphics[width=0.6\linewidth]{Images/schema amalitico.png}
    \caption{Esploso analitico con elementi truss, forze puramente assiali}
    \label{fig:esploso_an}
\end{figure}

Si utilizza una formulazione analitica, quindi dato lo schema delle forze nell'esploso in figura \ref{fig:esploso_an} si possono ricavare la matrice in figura \ref{fig:matr} del bilancio delle forze ai nodi, dalla quale si ricavano poi le azioni interne e le reazioni vincolari:

$$F_1 = -428\space N \quad F_2 = 428 \space N\quad F_3 = -428 \space N \quad F_4 = 428 \space N \quad F_5 = 428 \space N$$
$$F_6 = -856 \space N\quad F_7 = 0 \space N\quad R_{1x} = -428\sqrt3\space N \quad R_{1y} = -428 \space N\quad R_{2x} = -428\sqrt3 \space N$$

\begin{figure}[H]
    \centering
    \begin{gather*}
    \left\{
        \begin{array}{@{}c r r r r r r r r r r @{=} c}
P & & & & & & & & & & &428\\
P &+\frac{1}2 F_1 & -\frac12F_2 & & & & & & & & & 0\\
& +\frac{\sqrt3}{2}F_1 & +\frac{\sqrt 3}{2}F_2 & & & & & & & & & 0\\
& & +\frac{1}{2}F_2 & +F_3 & + \frac{1}{2}F_4& & & & & & & 0\\
& & +\frac{\sqrt{3}}{2}F_2 & & -\frac{\sqrt{3}}{2}F_4& & & & & & & 0\\
& +\frac{1}{2}F_1 & & +F_3 & & +\frac{1}{2}F_5 & -\frac{1}{2}F_6& & & & & 0\\
& \frac{\sqrt{3}}{2}F_1 & & & &-\frac{\sqrt{3}}{2}F_5 & -\frac{\sqrt{3}}{2}F_6& & & & & 0\\
& & & & & & \frac{\sqrt{3}}{2}F_6 & & +R_{1x}& & & 0\\
& & & & \frac{\sqrt{3}}{2}F_4 & + \frac{\sqrt{3}}{2}F_5 & & & & & +R_{2x} & 0\\
& & & & \frac{1}{2} F_4 & -\frac{1}{2}F_5 & & -F_7 & & & & 0\\
& & & & & & +\frac{1}{2}F_6 & +F_7 & & -R_{1y} & & 0
\end{array}
    \right.
    \end{gather*}
    \caption{Matrice del bilancio delle forze per il caso con forze solo assiali}
    \label{fig:matr}
\end{figure}


\subsection{Simulazione ad elementi finiti}
\label{s:truss_sim}%
Il modello Abaqus è stato costruito rispecchiando quello analitico risolto in precedenza (utilizzando quindi elementi T2D2), al fine di verificarne la validità. Questo verrà successivamente modificato per ridurre le semplificazioni e avvicinarsi al fenomeno reale; in figura \ref{fig:assembly_truss} il modello semplice in Abaqus

\begin{figure}[H]
    \centering
    \includegraphics[width=0.6\linewidth]{assembly_truss.png}
    \caption{Modello con elementi truss in Abaqus}
    \label{fig:assembly_truss}
\end{figure}

Per confrontare la simulazione col modello analitico si sfruttano i grafici che seguono:
\begin{itemize}
    \item In figura \ref{fig:freccine_truss} è rappresentato il verso delle azioni interne
    \item In figure \ref{fig:truss_grafichino} si può apprezzare il valore delle forze con segno, in base alla convenzione analitica in figura \ref{fig:esploso_an}
    \item In figura\ref{fig:truss_def} si osservano invece la struttura indeformata in nero con la sua deformata (fattore di scala $5e2$)
\end{itemize}

\begin{figure}[H]
    \centering

    \begin{minipage}{0.3\linewidth}
        \centering
        \includegraphics[width=\linewidth]{Images/truss_s11_arrows.png}
        \caption{Verso delle azioni interne}
        \label{fig:freccine_truss}
    \end{minipage}
    \hfill
    \begin{minipage}{0.3\linewidth}
        \centering
        \includegraphics[width=\linewidth]{Images/truss_grafico_columns.jpeg}
        \caption{Azioni interne sugli elementi con segno}
        \label{fig:truss_grafichino}
    \end{minipage}
    \hfill
    \begin{minipage}{0.3\linewidth}
        \centering
        \includegraphics[width=\linewidth]{Images/truss_def.png}
        \caption{Deformata e indeformata della struttura (scale factor: $5e2$)}
        \label{fig:truss_def}
    \end{minipage}

\end{figure}

I risultati del modello simulato convergono con quelli del modello analitico, confermandone la validità; si possono quindi estrarre i dati sulle deformazioni, mostrate in figura \ref{fig:truss_deform_ab}.

\begin{figure}[H]
    \centering
    \includegraphics[width=0.75\linewidth]{truss_deformazioni.png}
    \caption{Deformazioni nel modello a truss di Abaqus}
    \label{fig:truss_deform_ab}
\end{figure}

\section{Modellazione con elementi beam}
\label{ch:mod_beam}%
Ora che il modello semplificato è stato validato, si sfruttano le capacità di calcolo del software Abaqus per fare un modello più realistico, usando elementi di tipo beam, capaci di trasmettere azioni interne assiali e tangenziali ai nodi (con codice Abaqus B21)

In figura \ref{fig:beam_seed} si vede l'assembly della struttura con i vari nodi che compongono la mesh.

\begin{figure}[H]
    \centering
    \includegraphics[width=0.5\linewidth]{Images/beam_seed.png}
    \caption{Mesh del caso beam}
    \label{fig:beam_seed}
\end{figure}

A seconda del serraggio o meno dei bulloni che creano il vincolo al nodo $3$, si possono creare dei modelli simili dove però le travi coinvolte sono incernierate (bulloni non serrati) o incastrate (bulloni serrati). Entrambi i casi sono trattati di seguito.

\subsection{Travi incastrate}
\label{s:full_encastre}%

\begin{figure}[H]
    \centering

    \begin{minipage}{0.45\linewidth}
        \centering
        \includegraphics[width=0.5\linewidth]{Images/beam_ref_points.png}
        \caption{Assembly della struttura usando con modelli beam}
        \label{fig:ass_beam}
    \end{minipage}
    \hfill
    \begin{minipage}{0.45\linewidth}
        \centering
        \includegraphics[width=0.5\linewidth]{Images/nodo_2_zoom.png}
        \caption{Primo piano del nodo centrale}
        \label{fig:nodo_2_zoom}
    \end{minipage}
\end{figure}

In figura \ref{fig:ass_beam} si vede la struttura costruita con modello beam per le assi. Per la trattazione a travi incastrate, viene impostato un vincolo cinematico tra i punti $RP-11, RP-13, RP-14, RP-15, RP-18$ di incastro, bloccando spostamento e rotazione.
In figura \ref{fig:nodo_2_zoom} si vede un primo piano dei nodi sopracitati.

Avviata la simulazione, si possono leggere i risultati:
\begin{itemize}
    \item In figura \ref{fig:beam_enc_def} le deformazioni nelle posizioni degli estensimetri
    \item In figura \ref{fig:beam_az_int} le azioni interne sulle aste 
\end{itemize}


\begin{figure}[H]
    \centering

    \begin{minipage}{0.7\linewidth}
        \centering
    \includegraphics[width=1\linewidth]{Images/beam_enc_def.png}
    \caption{Deformazioni del caso beam incastrato}
    \label{fig:beam_enc_def}
    \end{minipage}
    \hfill
    \begin{minipage}{0.7\linewidth}
        \centering
    \includegraphics[width=1\linewidth]{Images/beam_az_int.png}
    \caption{Azioni interne delle aste del caso beam incastrato}
    \label{fig:beam_az_int}
    \end{minipage}
\end{figure}





Le immagini contenenti la deformata hanno uno scale factor di $2.767e+02$. La figura \ref{fig:beam_az_int} contenente le azioni interne all'asta è di particolare interesse: sono presenti azioni di trazione e compressione, coerenti con un modello di sforzi normali \textit{"a farfalla"} per la flessione, in grande contrasto con il modello visto precedentemente che permetteva solo trazione e compressione totale. In figura \ref{fig:profilo_farf_beam_primopiano} è rappresentato un primo piano delle azioni interne su un'asta per facilità di visione

\begin{figure} [H]
    \centering
    \includegraphics[width=0.5\linewidth]{Images/profilo_farf_beam_primopiano.png}
    \caption{Primo piano delle azioni interne su un'asta beam in flessione}
    \label{fig:profilo_farf_beam_primopiano}
\end{figure}

In figura \ref{fig:beam_enc_s_mises} invece si apprezzano gli sforzi equivalenti di \textit{Von Mises} 

\begin{figure}[H]
    \centering
    \includegraphics[width=0.7\linewidth]{Images/beam_enc_s_mises.png}
    \caption{Sforzo equivalente di Von Mises sulle aste}
    \label{fig:beam_enc_s_mises}
\end{figure}

\subsection{Travi incernierate}
\label{s:beam_2}%
Lo stesso modello può essere studiato nel caso di bulloni non serrati al nodo centrale, ovvero quello rappresentato in figura \ref{fig:nodo_2_zoom}. In questo caso si applica quindi un vincolo che permette la rotazione.

Applicando quindi il vincolo differente, e lasciando il resto invariato, si procede alle rappresentazioni e ai risultati:

\begin{itemize}
    \item In figura \ref{fig:beam_joint_def} le deformazioni alle posizioni degli estensimetri
    \item In figura \ref{fig:beam_joint_sm} gli sforzi equivalenti di Von Mises nelle aste
    \item In figura \ref{fig:beam_joint_s11} gli sforzi assiali ($S11$) nelle aste
\end{itemize}

\begin{figure}[H]
    \centering

    \begin{minipage}{0.7\linewidth}
        \centering
    \includegraphics[width=0.85\linewidth]{Images/beam_joint_def.png}
    \caption{Deformazioni del caso beam incernierato}
    \label{fig:beam_joint_def}
    \end{minipage}
    \hfill
    \begin{minipage}{0.7\linewidth}
        \centering
    \includegraphics[width=0.85\linewidth]{Images/beam_joint_sm.png}
    \caption{Sforzo di Von Mises nella struttura beam incernierata}
    \label{fig:beam_joint_sm}
    \end{minipage}
    \hfill
    \begin{minipage}{0.7\linewidth}
        \centering
    \includegraphics[width=0.9\linewidth]{beam_joint_s11.png}
    \caption{Sforzo assiale nella struttura beam incernierata}
    \label{fig:beam_joint_s11}
    \end{minipage}
\end{figure}

\chapter{Mensola bidimensionale intagliata con incastro}


\end{document}
